\subsection{Distribution strategy}
Distribution was achieved with \textit{MPI}  (Message Passing Interface).
It was, as planned, designed using a Master-Worker approach.
While the first design planned for a merging of all trees, testing proved it to be too heavy on both processing and communications.
That is why a different approach was used: each worker choses the \textit{n} more promising moves, and sends their winning ratio and number of simulations.
Thanks to this, results can be easily merged while only losing data on the moves that were not promising.
The \textit{n} factor is chosen in regards to the tested time of communications and merging, as well as on the game (simple games can have this \textit{n} be their total number of possible moves).

\subsection{Non-blocking calls}
With optimization in mind, non-blocking calls were used as often as possible. They allow the process to keep running while the message is not sent or received.
However, when the results of a non-blocking receive must be used (or when the buffer of a non-blocking send is reused or otherwise made inaccessible), it requires the use of a waiting function, to ensure that all the data has been correctly sent or received.
That means the time gained is comprised between the non-blocking call and this waiting function.