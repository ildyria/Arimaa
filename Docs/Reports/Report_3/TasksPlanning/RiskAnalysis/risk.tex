There will always be problems or delays that come up. The best way to deal with them is to think about what could go wrong in the project rather than waiting to get into deep water. The 5 kinds of risk are the following :
\begin{enumerate}
	\item Technical
	\item Resources
	\item Organisation
	\item Payments
	\item Suppliers/Purchases
\end{enumerate}
The main risk factor is the first item : technical. We are not dependent to anything such as purchases, ressources or organisation as we are a small group of workers.

To get into further details, here is a list of what might possibly become cumbersome:
\begin{itemize}
	\item getting used to the technology we will use (CAF, OpenMPI, OpenACC, OpenMP, Boost Library...).
	\item booking the use of \textit{Grid'5000}, we do not know how long is the waiting list in order to be able to use it.
	\item interoperability problems : most of us work on \textit{Windows} operating systems. However \textit{Grid'5000} runs on Linux, for this reason we will test our algorithm on a smaller scale. The computer science departement at INSA will be our first testing facility in order to make our cluster implemtation works.
\end{itemize}
In order to avoid theses difficulties, we made sure to dedicate enough time to study the technology used. We will also make sure to book \textit{Grid'5000} early. In the case where it would not be avialable, our tests would be done on a set of clusters from INSA's Computer Science department. While the computing power is not comparable to \textit{Grid'5000}, we still expect to get reliable results. A monthly test of our implementations in the Linux rooms of the department will guarantee us that our application does not have any interoperability problems.

Some other problems might come up later and we will try to deal with them as soon as possible to get the least delay as we possibly could.~\\
