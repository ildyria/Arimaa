This project is about developing an artificial intelligence for a board game using the \emph{Monte Carlo Tree Search Algorithm} and is coined as \emph{Fast \& Furious Game Playing, Monte Carlo Drift}. The board game chosen for this project is \emph{Arimaa}.
\newline

\emph{Arimaa} was conceived and developed in 2003 by Umar Syed, a computer science engineer. It was intentionally made difficult for computers to play, while following simple rules. Umar Syed offered a prize of 10,000 USD to the first program that can beat a human player in a game of 6 or more matches.
\newline

A revolution in scheduling algorithms originated in the \emph{Monte Carlo Tree Search} algorithm (\emph{MCTS}). \emph{MCTS} is a heuristic search algorithm used for making decisions. It concentrates on analyzing the optimum moves by expansion of a search tree of random samplings called the search space. In this project, the techniques of the \emph{MCTS} algorithm are utilized to find what move to make.
\newline

The program will be implemented using parallelisation techniques, so as to run it on different systems simultaneously. Eventually, the program will be executed on Grid'5000, a cluster of multi-core machines.
