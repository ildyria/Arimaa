\label{second part}
\subsection{The Minimax algorithm}
% miss the blank space
The Minimax algorithm is a way of finding an optimal move in a two player game. In the search tree for a two player game, there are two kinds of nodes, nodes representing ones moves and nodes representing the opponent's moves.\cite{graphics_minimax}
\begin{figure}[H]
\centering
	\begin{minipage}[b]{0.45\linewidth}
		\centering
		\includegraphics[height=1cm]{2_State_of_the_art/Arimaa_on_MCTS_Benoit/img/max.png}
		\caption{\label{fig:max}Nodes representing ones moves are generally drawn as squares, these are also called \emph{MAX} nodes.}
	\end{minipage}%
	\hspace*{1cm}
	\begin{minipage}[b]{0.45\linewidth}
		\centering
		\includegraphics[height=1cm]{2_State_of_the_art/Arimaa_on_MCTS_Benoit/img/min.png}
		\caption{\label{fig:min}Nodes representing the opponent's moves are generally drawn as circles, these are also called \emph{MIN} nodes.}
	\end{minipage}%
\end{figure}

The goal of a \emph{MAX/MIN} node is to maximize/minimize the value of the subtree rooted at that node. To do this, a \emph{MAX/MIN} node chooses the child with the greatest/smallest value, and that becomes the value of the \emph{MAX/MIN} node.

Note that it is typical for two player games to have different branching factors at each node. The move one makes could have an impact on what moves are possible for the opponent. In this example, one is ignoring what the game is in order to focus on the algorithm.

\begin{figure}[H]
\centering
	\begin{minipage}[b]{0.45\linewidth}
		\centering
		\includegraphics[height=1.5cm]{2_State_of_the_art/Arimaa_on_MCTS_Benoit/img/Minimax2.png}
		\caption{\label{fig:Minimax2}At the start of the problem, Minimax checks the single present node.}
	\end{minipage}%
	\hspace*{1cm}
	\begin{minipage}[b]{0.45\linewidth}
		\centering
		\includegraphics[height=3cm]{2_State_of_the_art/Arimaa_on_MCTS_Benoit/img/Minimax3.png}
		\caption{\label{fig:Minimax3}It begins like a depth first search, generating the first child.}
	\end{minipage}%
\end{figure}

So far, no evaluation values has been seen. The way Minimax works is to go down a specified number of full moves (where one \emph{full move} is actually a move by each player), then calculate the evaluation values for states at that depth. For this example, the tree will be explored one move down, which is one more level.
\begin{figure}[H]
\centering
	\begin{minipage}[b]{0.45\linewidth}
		\centering
		\includegraphics[height=3cm]{2_State_of_the_art/Arimaa_on_MCTS_Benoit/img/Minimax4.png}
		\caption{\label{fig:Minimax4}The values for those nodes are generated.}
	\end{minipage}%
	\hspace*{1cm}
	\begin{minipage}[b]{0.45\linewidth}
		\centering
		\includegraphics[height=3cm]{2_State_of_the_art/Arimaa_on_MCTS_Benoit/img/Minimax5.png}
		\caption{\label{fig:Minimax5}It chooses the minimum of the two child node values, which is 3.}
	\end{minipage}%
\end{figure}

The \textit{MAX} node at the top still has two other children nodes that will be generated and searched.

\begin{figure}[H]
\centering
	\begin{minipage}[b]{0.45\linewidth}
		\centering
		\includegraphics[height=3cm]{2_State_of_the_art/Arimaa_on_MCTS_Benoit/img/Minimax6.png}
		\caption{\label{fig:Minimax6}Since there is only one child, the \textit{MIN} node must take its value.}
		\end{minipage}%
	\hspace*{1cm}
	\begin{minipage}[b]{0.45\linewidth}
		\centering
		\includegraphics[height=3cm]{2_State_of_the_art/Arimaa_on_MCTS_Benoit/img/Minimax8.png}
		\caption{\label{fig:Minimax8}The third \textit{MIN} node chooses the minimum of its child node values, 1.}
	\end{minipage}%
\end{figure}

Finally, all values of the children of the \textit{MAX} node are at the top level, so it chooses the maximum of them, 15, and we get the final solution. 

\begin{figure}[H]
\centering
	\begin{minipage}[b]{1\linewidth}
		\centering
		\includegraphics[height=3cm]{2_State_of_the_art/Arimaa_on_MCTS_Benoit/img/Minimax9.png}
		\caption{\label{fig:Minimax9}Final tree.}
	\end{minipage}%
\end{figure}

The result of the algorithm is that the best move corresponds to the middle  \textit{MIN} node, since it will lead to the best possible state for us one full move down the road.

\subsection{The \ensuremath{\alpha\beta} pruning} %pruning = elaguage en anglais
The \ensuremath{\alpha\beta} method is a heuristic that decreases the number of leaves that will be explored by the Minimax algorithm. That way, the size of the tree will be smaller, the algorithm will be able to dive further and the time spend on more interesting subtree is greater.\\
If the leaf's position is less interesting than its parents, the algorithm will not explore anyfurther.

\subsection{Monte Carlo Tree Search Algorithm}
\subsubsection{Introduction}
Monte Carlo Tree Search (MCTS) is an algorithm used for taking decisions in artificial intelligence (AI) problems such as solving games or decision making in project management. It is based on generating a big number of random simulations in order to get reliable datas. To make such simulations, the program play the moves randomly for each players. Once it reaches a conclusion ( \textit{win} or  \textit{loss}), the program computes the statistics to get the odds of winning.
\subsubsection{How does it works ?}
The Algorithm creates a tree with all possible solutions with a small depth.
Then it starts to run random simulations starting from the leaves, to test the odds of the outcome.
Once the results are numerous enough (time based simulations are usually used), the decision is made depending on the odds of each subsequent leaves.
\subsubsection{Example}
MCTS algorithm relies on generating a great number of simulations. The following example has been simplified :
\label{sec:example}
\begin{figure}[H]
\centering
\includegraphics[height=5cm]{1_Presentation/1.2_Algorithm_MCTS_Benoit/img/schema.png}
\caption{\label{fig:schema}Explanation of the following figures.}
\end{figure}

\begin{figure}[H]
\centering
	\begin{minipage}[b]{0.45\linewidth}
		\centering
		\includegraphics[height=4cm]{1_Presentation/1.2_Algorithm_MCTS_Benoit/img/1.png}
		\caption{\label{fig:1}Run a first simulation from the root, get a favorable outcome (will be considered as a \textit{win}).}
	\end{minipage}%
	\hspace*{1cm}
	\begin{minipage}[b]{0.45\linewidth}
		\centering
		\includegraphics[height=4cm]{1_Presentation/1.2_Algorithm_MCTS_Benoit/img/2.png}
		\caption{\label{fig:2}Create a first leaf at depth 1 and run the simulation, get an unfavorable outcome (considered as a \textit{loss}).}
	\end{minipage}%
\end{figure}

\begin{figure}[H]
\centering
	\begin{minipage}[b]{0.3\linewidth}
		\centering
		\includegraphics[height=4cm]{1_Presentation/1.2_Algorithm_MCTS_Benoit/img/3.png}
		\caption{\label{fig:3}Create a second leaf at depth 1 and run the simulation (\textit{win}).}
	\end{minipage}%
	\hspace*{1cm}
	\begin{minipage}[b]{0.3\linewidth}
		\centering
		\includegraphics[height=4cm]{1_Presentation/1.2_Algorithm_MCTS_Benoit/img/4.png}
		\caption{\label{fig:4}Create a third leaf at depth 1 and run the simulation (\textit{loss}).}
	\end{minipage}%
	\hspace*{1cm}
	\begin{minipage}[b]{0.3\linewidth}
		\centering
		\includegraphics[height=4cm]{1_Presentation/1.2_Algorithm_MCTS_Benoit/img/5.png}
		\caption{\label{fig:5}Create a fourth leaf at depth 1 and run the simulation (\textit{win}).}
	\end{minipage}%
\end{figure}

Right now the odds of winning are 3/5. Now,  all the possible outcomes at depth 1 have been tested, the tree will be expand on the favorable leaves (here the second and fourth).

\begin{figure}[H]
\centering
	\begin{minipage}[b]{0.45\linewidth}
		\centering
		\includegraphics[height=4cm]{1_Presentation/1.2_Algorithm_MCTS_Benoit/img/6.png}
		\caption{\label{fig:6}Create a leaf at depth 2 with parent the second leaf at depth 1 and run the simulation (\textit{loss}), update the odds value of the node and making it less interesting than the fourth node. The algorithm will now work on the fourth node.}
	\end{minipage}%
	\hspace*{1cm}
	\begin{minipage}[b]{0.45\linewidth}
		\centering
		\includegraphics[height=4cm]{1_Presentation/1.2_Algorithm_MCTS_Benoit/img/7.png}
		\caption{\label{fig:7}Create a leaf at depth 2 with parent the fourth leaf at depth 1 and run the simulation (\textit{loss}), update the odds value of the node and making it as interesting as the second node. The algorithm will now work on the second node.}
	\end{minipage}%
\end{figure}
\begin{figure}[H]
\centering
	\begin{minipage}[b]{0.45\linewidth}
		\centering
		\includegraphics[height=4cm]{1_Presentation/1.2_Algorithm_MCTS_Benoit/img/8.png}
		\caption{\label{fig:8}Create a second leaf at depth 2 with parent the second leaf at depth 1 and run simulation (\textit{win}), update the odds value and continue to develop this leaf.}
	\end{minipage}%
	\hspace*{1cm}
	\begin{minipage}[b]{0.45\linewidth}
		\centering
		\includegraphics[height=4cm]{1_Presentation/1.2_Algorithm_MCTS_Benoit/img/9.png}
		\caption{\label{fig:9}Create a third leaf at depth 2 with parent the second leaf at depth 1 and run simulation (\textit{loss}) and update the odds value and switch to the fourth leaf to balance the search.}
	\end{minipage}%
\end{figure}
\noindent
Continue the Algorithm until a decent amount of simulation are run and/or the time limit is reached.
\begin{figure}[H]
\centering
	\begin{minipage}[b]{0.33\linewidth}
	\centering
		\includegraphics[height=4cm]{1_Presentation/1.2_Algorithm_MCTS_Benoit/img/10.png}
	\end{minipage}%
	\begin{minipage}[b]{0.33\linewidth}
	\centering
		\includegraphics[height=4cm]{1_Presentation/1.2_Algorithm_MCTS_Benoit/img/11.png}
	\end{minipage}%
	\begin{minipage}[b]{0.33\linewidth}
	\centering
		\includegraphics[height=4cm]{1_Presentation/1.2_Algorithm_MCTS_Benoit/img/12.png}
	\end{minipage}%
\end{figure}

Make a decision: the fourth leaf is chosen.\\
\newpage
\subsubsection{How to select the leaves to develop ?}
In the previous example, the leaves without winning rate were not expanding. But depending on the results of the simulations, wins can vary greatly. Therefore, more simulations will be run, on each leaf before choosing the ones to develop. For practical purpose, we will select the leaves to expand that has the highest value of the cost function UCT (Upper Confidence Bound 1 applied to Trees).\\
\bigskip
\begin{minipage}[b]{1\linewidth}
\centering
\begin{equation*}
f = \frac{w_i}{n_i} + c\sqrt{\frac{\ln t}{n_i}}
\end{equation*}
\medskip
\textit{UCT function}\cite{formula_UCT}

\end{minipage}%
\bigskip
\begin{itemize}
  \item \ensuremath{w_i}: number of wins after the \ensuremath{ i^{th}} node
  \item \ensuremath{n_i}: number of simulations after the \ensuremath{ i^{th}} node
  \item \ensuremath{c}: exploration parameter – theoretically equal to \ensuremath{\sqrt{2}} but in practice chosen empirically
  \item \ensuremath{t}: total number of simulations in a given tree node, equal to the sum of all \ensuremath{n_i}
\end{itemize}

If a leaf is too developed, the development of the nodes will not be worth it. Furthermore, a leaf with low winrate is not completely forgotten.\\

\subsubsection{Why using the Monte Carlo Tree Search ?}
 The advantage of MCTS with its basic form is that ones does not need to implement functions to improve the researches. Based on its random simulations, it will determine by itself which are the good options and which are not.\\ The more simulations are run, the more accurate the results will be.

\subsubsection{How much power is needed ?}
The more possible moves the game has, the more computing power it is required to solve it. In order to get decisions, it needs to go deeper in the tree and to search enough leaves. If the time or number of simulations is not sufficient, the algorithm might miss some important branches and fail to give plausible results. Therefore in order to get decent results, using high-end computer is mandatory, it allows us to get access to multi-threading technology in order to parallelize the simulations.